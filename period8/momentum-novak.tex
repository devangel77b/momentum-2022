\documentclass[aps,prl,preprint]{revtex4-1}

\usepackage{amsmath,amssymb,amsfonts}
\usepackage{siunitx}
\sisetup{separate-uncertainty=true}
\usepackage{fancyref}
\usepackage{graphicx}
\usepackage{listings}
\lstset{basicstyle=\ttfamily}
\usepackage{hyperref}
\usepackage[dvipsnames]{xcolor}
\hypersetup{
  colorlinks=true,
  linkcolor=violet,
  urlcolor=blue,
  citecolor=blue}

\usepackage{lineno}

\begin{document}
\linenumbers
\title{Ryan Novak has no title}
\author{Ryan Novak}
\email{rnovak\_25@mbs.net}
\author{D Evangelista}
\affiliation{Morristown-Beard School}

\newcommand{\student}[1]{{\textcolor{red}{#1}}}

\begin{abstract}
Momentum is the product of mass times velocity ($\vec{p}=m\vec{v}$), and kinetic energy of a moving mass is given by $\frac{1}{2}mv^2$. Both are potentially useful for understanding the mechanics of collisions, but while previous studies have examined momentum conservation during inelastic collisions \cite{ortega-2021-momentum}, they have not considered energy conservation during elastic collisions. \student{We examined the energy conserved in an elastic collision between 2 karts on a track. This led us to find that the energy lost was \qty{0.01\pm0.2}{\joule}, which was no significant difference from 0. This is important because they need this information to work and solve other things. 
}\end{abstract}
\maketitle

\section{Introduction}
\cite{ortega-2021-momentum} and others established that linear momentum ($p=m\vec{v}$) is conserved during inelastic collisions. Momentum reflects that the ``hitting power'' of an object involves both its mass and its velocity; here we consider one-dimensional (1D) collisions and will drop the vector notation for simplicity.  

On the other hand, kinetic energy for a point mass is given by:
\begin{equation}
\text{kinetic energy}\ E = \frac{1}{2} m v^2,
\label{eq:energy}
\end{equation}
where $m$ is mass in \si{\kilo\gram} and $\vec{v}$ is the velocity in \si{\meter\per\second} \cite{duchatelet-1741-reponse, coriolis-1829-calcul}. Energy reflects the potential to do work, and can be traded among many different forms. 

While simple linear momentum conservation was sufficient to predict velocity during inelastic collisions, energy could potentially allow us to understand the differences between  inelastic collisions, elastic collisions (the bodies bounce off one another), and explosions (when two stationary bodies shoot off in different directions). If energy is conserved during such collisions, it could be a powerful and useful tool in understanding.

Therefore, we wish to answer the question, is energy conserved during an elastic collision? We consider the total kinetic energy of the two-body system before ($E_0$) and after ($E_f$) the collision as
\begin{align}
E_0 &= \frac{1}{2} m_1 v_{1,0}^2\ \text{before collision}\\
E_f &= \frac{1}{2} m_1 v_{1,f}^2 + \frac{1}{2} m_2 v_{2,f}^2\ \text{after collision}.
\end{align}
We hypothesize that energy is conserved during such collisions:
\begin{equation}
H_0: E_f = E_0\ \text{(energy is conserved)}.
\end{equation}
Alternatively, energy could somehow increase during a collision; or it could decrease owing to friction or other loss mechanisms bleeding energy from the system: 
\begin{align}
H_1: E_f &> E_0\ \text{(energy increases)} \\
H_2: E_f &< E_0\ \text{(energy decreases)} .
\end{align}

We tested these hypotheses by conducting a large number of two-body elastic collisions with known masses and observing the velocities before and after the collision. For each collision, we found the total system $E_0$ and $E_f$ to determine whether energy is conserved or not. 

\section{Methods and materials}
\subsection{Elastic collision tests}
Elastic collision tests ($n=80$) were conducted using a \SI{2.1}{\meter} aluminum momentum test track (Vernier; Beaverton, OR; see \fref{fig:methods1}). The test track was outfitted with two small wheeled carts (mass of cart \SI{0.3}{\kilo\gram}) with simple sliding contact bearings; mass of each cart could be increased by addition of up to four \SI{0.125}{\kilo\gram} masses for a total mass of \SI{0.8}{\kilo\gram}. The mass was measured using an electronic balance (Ohaus; Parsippany, NJ).    Magnets of opposite polarity allowed the carts to stick after the inelastic collision. Carts were actuated by hand to provide initial velocities. The position of the two carts was measured using two ultrasound range sensors (Go Direct; Vernier; Beaverton, OR). Sensor data were logged at \SI{20}{\hertz} sampling frequency, via a wireless Bluetooth link, using the Graphical Analysis app (Vernier; Beaverton, OR) running on an iPad Air (Apple; Cupertino, CA). From the position data, the slope of position versus time was used to estimate velocities before and after the collision.  The mass, velocity, and direction down the track were varied haphazardly.   
\begin{figure}[h]
\begin{center}
\includegraphics[width=\columnwidth]{../figures/fig1/IMG_8061_cropped.jpg}
%\includegraphics[width=\columnwidth]{IMG_6253.jpg}
%\includegraphics[width=\columnwidth]{figures/fig1.png}
\end{center}
\caption{Momentum test track used for experiments. Total length \SI{2.1}{\meter}. }
\label{fig:methods1}
\end{figure}

 \subsection{Energy calculations}
 Calculations were performed using both Graphical Analysis and Sheets (Google; Mountain View, CA). 
 Energy before the collision was calculated as
 \begin{equation}
E_0 = \frac{1}{2} m_1 v_{1,0}^2
 \end{equation}
Similarly, the final energy was calculated as
 \begin{equation}
E_f = \frac{1}{2} m_1 v_{1,f}^2 + \frac{1}{2} m_2 v_{2,f}^2
 \end{equation}
and the difference in energy as 
 \begin{equation}
\Delta E =  E_f - E_0. 
 \end{equation}
For a nondimensional comparison, we also computed the ratio of the change of energy to the initial energy ($\hat{\pi}=\frac{\Delta E}{E_0}$). Additional plots and $t$-tests were done using R \cite{r-2021} and the \lstinline{tidyverse} library \cite{wickham-2019-welcome}. 

\section{Results}
\Fref{fig:results1} shows the position data for a typical elastic collision with equal masses, $m_1=m_2=\qty{0.3}{\kilo\gram}$. Before the collision, $m_2$ is stationary while $m_1$ approaches with velocity $v_1=\qty{0.3}{\meter\per\second}$. The masses collide at $t=\qty{2.1}{\second}$. After the collision, $m_1$ is stopped, while $m_2$ is moving with a velocity $v_{2,f}=\qty{0.3}{\meter\per\second}$. 
\begin{figure}
\begin{center}
\includegraphics{../figures/fig2/elastic.pdf}
\end{center}
\caption{Position data for a typical elastic collision with equal masses, $m_1=m_2=\qty{0.3}{\kilo\gram}$. Initially, $m_2$ is stationary and $m_1$ is moving at $v_{1,0}=\qty{0.3}{\meter\per\second}$. The masses collide at $t=\SI{2.1}{\second}$. After the collision, $m_2$ is moving with velocity $v_{2,f}=\qty{0.3}{\meter\per\second}$. }
\label{fig:results1}
\end{figure}

\Fref{fig:results2} shows a bar plot of the change in energy during all collisions, $\Delta E = \qty{-0.01\pm0.02}{\joule}$ (mean$\pm$s.d.). The mean $\Delta E$ is not significantly different from \qty{0}{\joule} (one-sample $t$-test, n=80, d.f.=79, p=0.2596). 
\begin{figure}
\begin{center}
\includegraphics{../figures/fig3/energy_change.pdf}
\end{center}
\caption{Change in energy ($\Delta E$) during all elastic collisions. $\Delta E = \qty{-0.01\pm0.02}{\joule}$ (mean $\pm$ s.d.). The mean $\Delta E$ is not significantly different from \qty{0}{\joule} (one-sample $t$-test, n=80, d.f.=79, p=0.2596).}
\label{fig:results2}
\end{figure}

\section{Discussion}
\student{The results of this lab show that energy is conserved when two cars make contact with each other on a track. The energy lost was \qty{0.1\pm0.2}{\joule}. These results mean that barely any energy was lost, confirming that energy is conserved in an elastic collision. This can be applied to other elastic collisions in basketball when you miss a shot and it bounces off the metal rim. You can use these equations to find the energy-momentum and velocity before and after a collision in situations you care about.}

\section{Acknowledgements}
We thank \student{L.~Koepff, L.~Atkins}, C.~Payette, J.~Bartholomew, B.~Turner, K.~Mauger, and S.~McCormick for useful advice and comments on the manuscript, and A.~Hahn, S.~Kealy, K~Muttick, and A.~Sankar, for advice on the design of experiment and learning objectives. Major lab equipment was gifted by the Alumni Association at Morristown-Beard School. This work was funded by ONR \#N123456789, AFOSR, NIH, and the National Science Foundation. 

\bibliographystyle{apsrev4-1} % Tell bibtex which bibliography style to use
\bibliography{../references/momentum} % Tell bibtex which .bib file to use (this one is some example file in TexLive's file tree)

\end{document}